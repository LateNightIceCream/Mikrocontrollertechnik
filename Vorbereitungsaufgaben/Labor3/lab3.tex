\documentclass[a4paper, 12pt]{article}

\input{../preamble}

%%%%%%%%%%%%%%%%%%%%%%%%%%%%%%%%%%%%%

\begin{document}

%%%%%%%%%%%%%%%%%%%%%%%%%%%%%%%%%%%%%
  \includepdf{./titlepage/titlepage.pdf}
  \clearpage
  \setcounter{page}{1}
%%%%%%%%%%%%%%%%%%%%%%%%%%%%%%%%%%%%%

\section{Konfiguration des MSP430 zur Port-Ein- und Ausgabe}
Jeder Port besteht aus 8 physischen Pins (Px.0 bis Px.7) und hat entsprechend mehrere 8-bit Register, die zur
Konfiguration dienen. Die für die Ein- und Ausgabefunktion relevanten Register
sind:
\begin{enumerate}
\item[] \textbf{PxSEL}
  \begin{enumerate}
    \item[] \textit{Function Select Register}: Legt fest, welche Funktion verwendet wird.
      \begin{enumerate}
        \item[0:]  I/O-Funktion
        \item[1:]  Peripherie-Funktion
      \end{enumerate}
  \end{enumerate}
\item[] \textbf{PxDIR}
  \begin{enumerate}
  \item[] \textit{Direction Register}: Legt die Richtung, also Ein- oder Ausgabe fest.
  \begin{enumerate}
    \item[0:]  Input
    \item[1:]  Output
  \end{enumerate}
  \end{enumerate}
\item[] \textbf{PxOUT/PxIN}
  \begin{enumerate}
\item[]    \textit{Input/Output Register}: Bestimmen, abhängig vom Wert des PxDIR, den
   eigentlichen Ein- bzw. Ausgabewert.
 \end{enumerate}
\item[] \textbf{PxREN}
  \begin{enumerate}
    \item[] \emph{Resistor ENable Register}: Legt fest, ob Pullup- oder Pulldown
      Widerstände verwendet werden sollen (PxDIR auf Input).
      \begin{enumerate}
        \item[0:] Widerstand deaktiviert
        \item[1:] Der Wert in \emph{PxOUT} entscheidet ob Pullup/down
          \begin{enumerate}
            \item[0:] Pulldown
            \item[1:] Pullup
          \end{enumerate}
      \end{enumerate}
  \end{enumerate}
\end{enumerate}

\section{Initialisierung eines LC-Displays}
\section{LCD-Anschluss und Konfiguration des MSP430}
\section{Zeichendarstellung}
  
 \end{document}

% Local Variables:
% TeX-engine: luatex
% End:
