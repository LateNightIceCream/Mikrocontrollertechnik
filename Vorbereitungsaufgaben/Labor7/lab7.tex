\documentclass[a4paper, 12pt]{article}

\input{../preamble}

%%%%%%%%%%%%%%%%%%%%%%%%%%%%%%%%%%%%%

\begin{document}

%%%%%%%%%%%%%%%%%%%%%%%%%%%%%%%%%%%%%
  \includepdf{./titlepage/titlepage.pdf}
  \clearpage
  \setcounter{page}{1}
%%%%%%%%%%%%%%%%%%%%%%%%%%%%%%%%%%%%%
  \section{Timer A Initialisierung}
  Timer A wird über das \inlinecode{TA}

  Entsprechend des Timermodus muss dann möglicherweise noch das \inlinecode{TACCR0} Register
  gesetzt werden.
  
  \section{Timer A Start}
  Der Timer wird über die \emph{Mode Control} (MC) Bits in einen Modus, d.h.
  up-, up-down-, bzw. Continuous-Mode versetzt und dadurch automatisch gestartet.  

  \section{Output Mode und Zählmodus}

  \section{Minimale und maximale Frequenz}
  Die minimale Frequenz die erzeugt werden kann, ist
  \[f_{min} = \frac{f_{CLK}/d}{2\cdot 65536}\]
  wenn \inlinecode{TACCR0} auf den
  Maximalwert (\inlinecode{0xFFFFFF}) gesetzt, der Taktteiler auf dem Maximalwert
  ($d=8$) und der Timer im up-down-Mode ist.\\

  Die maximale Frequenz, die erzeugt werden kann, ist die halbe
  Timertaktfrequenz im up-Mode
  \[f_{max} = \frac{f_{CLK}}{2}\]
  


 \end{document}

% Local Variables:
% TeX-engine: luatex
% End:
