\documentclass[a4paper, 12pt]{article}

\input{../preamble}

%%%%%%%%%%%%%%%%%%%%%%%%%%%%%%%%%%%%%

\begin{document}

%%%%%%%%%%%%%%%%%%%%%%%%%%%%%%%%%%%%%
  \includepdf{./titlepage/titlepage.pdf}
  \clearpage
  \setcounter{page}{1}
%%%%%%%%%%%%%%%%%%%%%%%%%%%%%%%%%%%%%

  \section{Wie ist ein C-Programm aufgebaut?}

  \subsection{Header.h}
Mittels Präprozessoranweisung (\inlinecode{#include}) können \emph{Header-Dateien} in
C-Programme eingebunden werden. Diese enthalten \inlinecode{#define} Anweisungen zur
Definition von Konstanten und Macros sowie Funktions- und Strukturprototypen
(Deklarationen) und erlauben somit eine modulare und damit übersichtliche
Strukturierung des Programmes. Außerdem können sie als einfache Dokumentation
dienen, da z.B. Ein- und Ausgaben einer Funktion direkt aus der Deklaration
erkennbar sind.

\subsection{Quellcode.c}
Quellcodedateien enthalten die eigentliche Programmlogik/den Programmablauf. Hierzu gehört die Datei \emph{main.c} sowie alle nötigen \emph{Funktionsdefinitionen}.

Innerhalb der \emph{main.c} werden erst alle Präprozessoranweisungen wie
\inlinecode{#include} oder \inlinecode{#define} gelistet. Als Einstiegspunkt des Programmes gilt dann
die \inlinecode{main()} Funktion, welche dem Betriebssystem oft den Integer-Wert $0$
zur Rückmeldung eines Fehlerfreien Ablaufes zurückgibt; Für Microcontroller ist
dies jedoch nicht von Bedeutung, da sie zum einen kein Betriebssystem besitzen
und zum anderen meist in einer Endlosschleife innerhalb der
\inlinecode{main()}Funktion laufen.


\vspace{\parskip}
\begin{lstlisting}
  #include <stdio.h>
  #define KONST 0
  
  int main() {
   ... 
    return 0;
  }
\end{lstlisting}

  \section{Was sind Assembler, Linker, Compiler?}

  \subsection{Assembler}

  \subsection{Linker}

  \subsection{Compiler}
  
  \section{Was ist eine Entwicklungsumgebung?}
  Eine Entwicklungsumgebung ist ein Programm, das mehrere Programme zur
  Softwareentwicklung, wie z.B. einen Programmcode-Editor, Syntax-Highlighting,
  Compiler, Linker, Debugger, GUI, etc., für eine oder mehrere Programmiersprachen bereitstellt.

  \section{Welche Aufgabe haben die Module eines C-Programms?}
  siehe 1: Header + C-dateien


  
\end{document}

% Local Variables:
% TeX-engine: luatex
% End:
